%%% Local Variables: 
%%% mode: latex
%%% TeX-master: t
%%% End: 

\chapter{总结与展望}

\section{总结}
本文针对角毛藻独特的生物形态特征以及其显微图像的特点,提出了一种灰度曲面方向角模型和支持向量机结合的角毛藻显微图像分割方法。本文将分割问题视为分类问题并使用支持向量机来完成分类,所得图像中的所有像素将被分为目标和背景两部分。灰度曲面方向角模型用于产生支持向量机的输入特征,连通域预分割结果用于产生支持向量机的训练样本。经过实验验证,本文的分割方法能够被成功运用在角毛藻显微图像数据集上并且取得了很好的效果。本文的主要内容如下:
\begin{enumerate}
\item 角毛藻显微图像的特征提取,通过分析角毛藻独特的生物形态特征以及显微图像的特点,利用灰度曲面方向角模型实现角毛藻显微图像的特征提取,使用该算法可以有效地提取角毛特征信息,获得比传统特征提取方法更为精确、完整的效果。
\item 连通域预分割,选择所提取特征图中含有较多角毛信息和细胞体边缘信息的两幅灰度特征图进行预分割操作,利用大津法和最大连通域填充处理所选特征图得到预分割结果。
\item 支持向量机分类和后续处理,使用支持向量机进行分类,训练过程中输入灰度映射特征和预分割结果,预测过程中输入原始待处理图像及其对应的特征图。分类结束后,通过选取最大连通域及一些形态学操作处理
分类结果图,最终得到角毛藻显微图像的分割结果。实验结果表明,本文的方法可以较为完整、精确地分割出角毛藻目标细胞及其角毛部分来。
\end{enumerate}
\section{展望}
本文提出的角毛藻显微图像分割的方法能够有效地达到预期的分割效果,但是仍然存在一些需要研究并改进的方面:
\begin{enumerate}
\item 虽然本文的方法可以提取出大量的角毛信息,但是角毛藻轮廓部分仍然存在一些没有填充完整的地方且角毛边缘不平滑,还需要进一步研究和处理。
\item 本文的方法只对单个角毛藻目标细胞分割有效,对于多个角毛藻细胞,无法得到令人满意的效果,需要继续改进和优化方法。
\end{enumerate}
